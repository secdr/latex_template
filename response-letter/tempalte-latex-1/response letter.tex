\documentclass{article}
\usepackage{a4wide}
\title{Response to the Reviewer's Comments}
\author{Johan van Benthem \and Jan van Eijck \and Barteld Kooi}
\begin{document}

\maketitle

The comments of the reviewers have been very useful, for they
made us aware that some major points we were trying to convey
remained obscure even to well-informed readers. Following up
on the reviewers' suggestions, we have restructured the paper
by changing the order of some sections, by putting more emphasis
on motivation, and by adding clarifications at crucial spots.

In addition, we have also corrected all factual errors and
minor infelicities of presentation that the reviewers brought
to our attention. Below we indicate for each individual comment
how we have dealt with it.

\begin{verbatim}

>> REFEREE REPORT 1<<

> Review of "Logics of Communication and Change"
> by van Benthem, van Eijck, Kooi
>
> This paper studies dynamic-epistemic logics for reasoning about the
> exchange of factual and higher-order information.  More precisely,
> the paper examines the extension of two existing logics with common
> knowledge, obtaining PAL-RC (a logic of public announcements) and LCC
> (a logic for general update actions). The driving goal is to obtain
> "reduction axioms" for common knowledge, which lead to a uniform
> proof technique for completeness and complexity analysis.
>
> My overall reaction to the paper is ambivalent. On the one hand, I am
> familiar with (some of) the work of the authors, and highly respect
> it. And this paper follows in the tradition of being a solid
> technical paper about dynamic-epistemic logics. On that account, I am
> leaning towards accepting the paper. On the other hand, the overall
> motivation and structure of the paper leaves a lot to be desired. My
> recommendation, therefore, is for the authors to revise the paper to
> clarify the contribution and the aim of the paper, prior to it being
> accepted. I do not have any comments on the technical aspects of the
> work. The parts that I checked work out, and the parts that I did not
> check carefully feel right. The comments that follow I hope will help
> the authors understand where I have gotten confused; if I did get
> confused, knowing the subject, I expect average reader might get lost.
\end{verbatim} 

We take the point, and we think our new version is much clearer.

\begin{verbatim}
> I have read the paper a few times now, and I am still a bit at a loss
> trying to describe the overall story. The main contribution seems to
> be a technical contribution in a fairly narrow sense (and the
> introduction agrees with my assessment): consider dynamic-epistemic
> logics with common knowledge, and look at their completeness proofs.
> These already exist (as the authors are well aware). However, their
> proofs are ugly. Well, not really. But they do not follow the elegant
> approach of using reduction axioms to translate common
> knowledge down to other operators - isn't that sad? Wonder if we can
> somehow extend the languages to restore a form of reduction axioms
> and thereby get more uniform completeness proofs? The authors answer
> is: yes, at least, in the case we have examined: PAL, and E-PDL.
>
> The obvious question at this point is: why? Completeness proofs for
> PAL + common knowledge seem to already exist, so it is already known
> that PAL + common knowledge is powerful enough to capture
> proof-theoretically all the valid properties of models. So there is
> no need to extend the language with a new operator to obtain
> completeness. The new operator is only needed to make the proofs
> nicer. That seems like a somewhat weak motivation. The authors seem
> to at least partially agree, as they point out that logics with model
> updates (which is the kind of operator that is added to the logic)
> are of independent interest, and that seems to justify the
> exploration of the extended PAL logic in the core of section 2. This
> muddles the story: a big part of section 2 has nothing to do with
> reduction axioms, the main argument put forth in the introduction. It
> is distracting. Please do not misunderstand me: it is technically
> interesting, and there are real results here. But that distract from
> the storyline nevertheless. Either move the results somewhere else,
> or change the story. (I would advocate the latter...)
\end{verbatim}

We read the objection of the reviewer as a complaint about the
story line. Our restructuring should have remedied this.

We now emphasize the real reason why we want the reduction axioms:
as a tool for {\em compositional analysis} of the epistemic effects of
many different types of information-carrying event. It is actually
surprising that this is possible. We see this compositional analysis
as the heart of dynamic epistemic logic, not just a negotiable
'luxury' for some cases. As a side-effect, such axioms
'reduce' a complete dynamic epistemic language to its static
epistemic base -- but this is a consequence of the analysis,
not the main motivation. Of course, as we now also stress, this
reduction does have another interesting interpretation by itself:
the static language must be rich enough to 'pre-encode' the
effects of the relevenat class of informational events. This
is a requirement on optimal language design which is also known
from other areas of logic, such as conditional logic vis-a-vis
belief revision. Finally, and we may have given the wrong
impression that *this* was our main point: the reduction
analysis tends to make for simpler completeness proofs,
and some direct borrowing of known properties of the
static base into the full dynamic system.

What we also find interesting is how this methodology puts us on
the track of new epistemic operators that have not been isolated
as important before, even though they were there just below the
surface. In particular, relativized common knowledge seems a
new notion of independent interest.

\begin{verbatim}
> Section 3 seems to be there to motivate the logic introduced in the
> following section. A bit odd to place it there - it really feels like
> this ought to be in the introduction. However, of course, this cannot
> be done, because there is the intervening section 2, which describes
> the "pilot case" of the real work that will be done in section 4.
> Again, section 2 does not seem to fit where it is.
>
\end{verbatim}

We have swapped Sections 2 and 3, with the intent of improving the
story line.

\begin{verbatim}
> Section 4 is the core of the paper. (And, I guess this is what the
> title of the paper refers to.) And here again, the story gets
> slightly muddled. At this point in reading the paper, one stills
> hangs on to the thread that the authors want to derive reduction
> axioms for common knowledge in various dynamic-epistemic logics.
\end{verbatim} 

But see the above.

\begin{verbatim} 
> And to do so, they now look at Epistemic PDL, which is of course just PDL
> where the programs are interpreted as syntactic description of the
> epistemic reachability relations. It is well-known that the standard
> axioms for PDL are complete for the standard interpretation, and here
> as well, the motivation for having reduction axioms seems to be to
> make the proofs easier. But this is not described at all. Instead of
> following up on this storyline, the authors switch gear and say:
> look, we want to be able to reason about changes (see example in the
> last section),
> and this calls for a more powerful logic. Well, no, not really, as it
> turns out, as the expressive power of LCC is the same as PDL (Theorem
> 12). So what is the point of LCC? To make the completeness proof
> easier, as the introduction seems to indicate? I would argue that the
> proof of completeness of PDL is not that complicated. But I agree
> that it does not follow directly from reduction axioms. Although,
> arguably, the derivation of the reduction axioms in this paper seem
> to be at an equivalent level of complexity. But the authors *seem* to
> have a further motivation to introduce LCC, but it is not spelled out
> clearly what that motivation is -- although I do believe there is one.
\end{verbatim} 

The referee is right: the completeness proof of PDL is not that
complicated. More than that: it is very elegant, and we have nothing
to add to it.

The new presentation makes our intentions clearer, we think.
The completeness proof of the logic of epistemic action models as
provided by Baltag, Moss and Solecki is complicated. Our method of
reduction axioms allows us to `lift' the elegant PDL completeness
proof to the logic of epistemic action models. And again: the
reduction axioms that we find are interesting in their own right
as we are after compositional analysis of epistemic effects of events
observed in groups of agents, and earlier dynamic epistemic logic were
simply unable to provide that methodology for common knowledge.

The `point' of LCC is to show that `really' epistemic PDL, contrary to
what the designs of intricate logics of communication and knowledge
suggest, is all that is needed for a rich logic of communication and
change. So if anything, this is a new vindication of PDL, rather than
a criticism of it.

It has to be said, though, that we do not provide a knock-down argument
why this full system is {\em needed} for our style of analysis. We have
not been able to find anything weaker that works, but our list of
open questions at the end does ask if other weaker 'solutions' exist.

\begin{verbatim} 
> Thus, after having read the paper a few times, I am left with an
> appreciation of a number of technical results that seem to be
> connected, but no clear idea exactly why those results are worth
> pursuing, and what the connections are exactly. I therefore strongly
> encourage the authors to focus the story and turn what seems like a
> number of forking meandering paths into a clear trail.
\end{verbatim}

We hope that our new version is much more focussed.

The author is right on one further point. We have chosen to add
events that change the world to the standard DEL-framework, since
we feel that this increases the scope with little effort, while
still operating with the same methodology. But it is clear that
this is orthogonal to our main points. We have considered taking
it out (but that seemed a pity), or to put it into a separate
section: but that would lead to too much duplication. Our solution
now is to leave the story combined, while pointing out to the reader
who so prefers that a 'change-free' reading of our main notions and
results can be obtained by merely omitting all our technical talk
about substitutions.

\begin{verbatim}
> As far as style is concerned, the work "perspicuous" appears three
> times on the first page; once in the abstract, and twice in the
> second paragraph. While perspicuous is a delicious word, it is rare
> enough that seeing it three times makes one frown.
\end{verbatim} 

We have rephrased things using a wider repertoire of adjectives.

\begin{verbatim}
> Also, typo on p.3,
> first paragraph of 2.1: "PLaza" instead of "Plaza".
\end{verbatim}

Corrected.

\begin{verbatim} 
>>REFEREE REPORT 2<<

> List of typos, corrections and comments:
>
> 1. One of the key definitions ...
\end{verbatim}

The referee is correct that this definition was wrong.

The following change solves the problem. We now take the transitive
closure rather than the *reflexive* transitive closure in the
definition. We have adapted the rest of the text including the
axioms accordingly, and now also the remark on page 5 is correct.

\begin{verbatim}
> 2. The above correction ...
\end{verbatim}

We have considered the suggestion of the referee to write $C(\psi|\phi)$
rather than $C(\phi,\psi)$, but found eventually that formulas become
less readable when they also contain $[\phi]$ operators.

However, we did take up the referee's suggestion to provide a
brief way of reading formulas with relativized common knowledge.

\begin{verbatim}
> 3. The two axioms called Dist (distribution) on page 5 are usually called
>    K.
\end{verbatim} 

This is indeed correct. We chose to give a more informative name
to these axioms for our purposes here. We added a remark to establish
the connection with standard terminology.

\begin{verbatim}
> 4. A typo
\end{verbatim}

Corrected.

\begin{verbatim}
> 5. In Def 6 (``closure'')
\end{verbatim}

We followed the referee's advice and added a remark.

\begin{verbatim}
> 6. Theorems 2 and 3 (in section 2.5) are not proved.
\end{verbatim}

We have added a remark on the well-definedness of the translation in
definition 9 by saying that a complexity measure that the referee
suggests can be given and that the induction can follow this measure.
We also briefly discuss the somewhat tangential role of the final
axiom for two consecutive epistemic actions.

\begin{verbatim}
> 7. The proof of Theorem 4 refers to Lemma 2 ...
\end{verbatim}

We have removed the reference to Lemma 2 and now state in the proof
what we intended to get from Lemma 2.

\begin{verbatim}
> 8. In the formulation of the EL-RC Game ...
\end{verbatim}

The referee is correct to point out the ambiguity. We have resolved
it by renaming 'n' to 'k'.

\begin{verbatim}
> 9. In defining the third kind of moves ...
\end{verbatim}

We have added a clarifying remark on complements.

\begin{verbatim}
> 10. No intuition for ``epistemic PDL'' (Section 4.2, Definition 21) is
>     provided.
\end{verbatim} 

We have added references and clarification. 

\begin{verbatim}
> 11. On page 31.
\end{verbatim}

We have corrected the text in accordance with the referee's comments.

\begin{verbatim}
>> REFEREE REPORT 3<<

> Comments:

> first bullet
\end{verbatim} 

Indeed, the proof of Lemma 1 was very terse.
We added more details to the proof.

\begin{verbatim} 
> second bullet
\end{verbatim}

The reason we look at EL-RC, is that it seems to be the poorest
logic (in terms of expressivity) for which reduction of common
knowledge can be achieved. In the paper it also has a didactical
purpose, since EL-RC is rather simple compared to full epistemic
PDL, making for a nice explanation of the key idea of language
extension by relativization, which may be harder to discern behind the
technicalities of our program transformers in Section 4.

\begin{verbatim}
> third bullet
\end{verbatim}

We have moved this section to anothe rplace, right after the introduction.

\begin{verbatim}
> fourth bullet
\end{verbatim}

We have replaced references to manuscripts with references
to published reports, and in once case to a URL for download.

\begin{verbatim}
> fifth bullet
\end{verbatim}

A reference to Heikki Tuomine's ECAI 88 paper has been added.
Indeed an interesting link that we were not aware of!

\begin{verbatim}
> Some corrections
> p.1, middle: Explain briefly the meaning of the [phi] operator here.
\end{verbatim}

We indicate how to read the formula now, including the [phi] operator.

\begin{verbatim}
> second bullet: The languages are assumed ...
\end{verbatim}

We assume throughout that the number of basic propositions and the
number of agents are finite, because it makes the model theoretic
points easier to make. We are not sure that allowing an infinite
number of proposition letters would affect expressivity. After all,
the number of agents and basic propositions mentioned in a single
formula will remain finite in any case, given the way we have set up
the languages. But we do use a logical finiteness lemma at some stage,
whose proof does depend on having only a finite stock of proposition
letters. Of course, working with finite update models is essential
to our reduction axioms, because otherwise, we would encounter  infinite
conjunctions and disjunctions in listing indistinguishable events.


\begin{verbatim}
> p.5, Remark: `...dynamic...':
\end{verbatim}

Corrected

\begin{verbatim}
> p.6, line 8
\end{verbatim}

We added further details to the proof.

\begin{verbatim}
> First paragraph of section 2.5: edit the sentence `Note that this holds
> ...'
\end{verbatim}

The sentence has been replaced by a hopefully less obscure one.

\begin{verbatim}
> p.8, the [phi] move
\end{verbatim}

Corrected

\begin{verbatim}
> p.9, Def 12 R: redundant pair of parentheses.
\end{verbatim}

Corrected

\begin{verbatim}
> p.9, line -9: what does it mean ...
\end{verbatim}

Corrected

\begin{verbatim}
> p.10, line -8: remove a ( and reverse the arrow in this definition.
\end{verbatim}

Corrected

\begin{verbatim}
> p.11-12, Def. 14, 15
\end{verbatim}
Corrected

\begin{verbatim}
> p.11, Lemma 4
\end{verbatim}

Corrected as min(x,y)-n

\begin{verbatim}
> p.12, Def. 15
\end{verbatim}

The definition should read `0 <n <= m', since we do not want any of the
lines to be empty. We now have different notation for the different models.

\begin{verbatim}
> p.13, Lemma 6
\end{verbatim}

The textbooks we consulted usually leave this as an exercise. To save
space we also do not give the proof, but now we do refer to some
precise page numbers.

\begin{verbatim}
> p.18 Def.18
\end{verbatim}

Reconciled. We now only refer to update models.

\begin{verbatim} 
> p.21 proof of Thm 9
\end{verbatim}

We replaced R with B.

\begin{verbatim} 
> p. 26, Def. 27
\end{verbatim}

A reminder now preceeds the definition.

\begin{verbatim} 
> p. 27, line -8
\end{verbatim}

We now say that the simple cases reduce to EL-RC.

\begin{verbatim} 
> p. 31, second last bulet: can you support `...we have a proof that...'
> with a reference?
\end{verbatim}

Our manuscript ``A fixed-point look at update logic'' is the reference. It
is only available so far as a downloadable document, whose URL is provided.

\begin{verbatim} 
> p. 31, line -2
\end{verbatim}

We have corrected the mistake.

\begin{verbatim} 
> p. 32, line 3 any reference here?
\end{verbatim}

A reference has been added.

\begin{verbatim} 
> p. 32, middle: I do not understand the sentence `In the setting of
> ... '. Edit, please.
\end{verbatim}

Edited

\end{document}



